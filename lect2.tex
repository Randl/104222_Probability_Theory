$$P(\Omega|A) = \frac{P(A \cap \Omega)}{P(A)} = \frac{P(A)}{P(A)} = 1$$
$$P(\emptyset|A) = \frac{P(A \cap \emptyset)}{P(A)} = \frac{P(\emptyset)}{P(A)} = 0$$

It is easy to show
$$P\left( \bigcup_{j=1}^\infty B_j | A \right) = \sum_{j=1}^\infty P\left( B_j | A \right) $$
Also 
$$P(B|A) = \frac{P(A\cap B)}{P(A)} \in [0,1]$$
\paragraph{Example}
$$A = \left\{ \parbox{2cm}{at least one of dices is 6 } \right\}$$
$$B = \left\{ \parbox{1cm}{sum is 7 } \right\}$$

Then
$$P(A) = \frac{11}{36}$$
$$P(B) = \frac{1}{6}$$
$$P(A\cap B) = \frac{1}{18}$$

$$P(A|B) = \frac{P(A\cap B)}{P(B)} = \frac{1}{3} $$
$$P(B|A) = \frac{P(A\cap B)}{P(A)} = \frac{2}{11} $$


\paragraph{Alternative form of conditional probability}
$$P(A\cap B) = P(A)P(B|A) = P(B)P(A|B)$$
\paragraph{Definition} $A$ and $B$ are independent if $$P(A\cap B) = P(A)P(B)$$
\subparagraph{Note} If $P(A) = 0 $ or $P(B) = 0$ then $A$ and $B$ are independent.
\paragraph{Theorem} Suppose $P(A) \neq 0$, $P(B)\neq 0$. Then three following properties are equivalent:
\begin{enumerate}
	\item $A$ and $B$ are independent
	\item $P(A|B) = P(A)$
	\item $P(B|A) = P(B)$.
\end{enumerate}
\subparagraph{Proof} is trivial from definition of conditional probability.
\paragraph{Definition} Independence of set of events $\left\{ A_j \right\}_{j=1}^n$ is
$$\forall I \subseteq \left\{ 1,2,\dots n \right\} \quad P\left( \bigcap_{i\in I} A_i \right) = \prod_{j \in I} P(A_j)$$

\paragraph{Note} Independence and disjointedness are different things:
$$P(A\cap B) = P(A)P(B)$$
$$P(A\cap B) = 0$$

\paragraph{Example}
$$A = \left\{ \parbox{2cm}{at least one of dices is 6 } \right\}$$
$$B =  \left\{ \parbox{2cm}{first is 6 } \right\} $$
$$C =  \left\{ \parbox{2cm}{second is 6 } \right\} $$
$$B\cap C =  A^C $$
$$P(A) = 1 - P\left(A^C\right) = 1 -\underbrace{ P(B)P(C)}_{\text{independent}} = 1 - \frac{25}{36} = \frac{11}{36}$$